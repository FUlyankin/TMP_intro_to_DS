
Хочу ручные задачки на работу основных пайплайнов из sklearn, чтобы убедиться что студенты понимают как они работают. 



- crossval 

Вариант хард:  Дана выборка 10 наблюдений. Разбили на 5 частей. Картинка с подцветкой как разбили. Обучить на каждой части линейную модель $y = \beta \cdot x.$ Посчитать прогнозы. Метрику качества на каждом фолде. Найти усреднённое качество. 

Варивант софт:  Даны уже обученные деревья. По ним строим прогнозы и вычисляем качество.  


- Gridsearch

Хард:  регрессия - сами учим, смотрим какая l2 регуляризация лучше. 

Софт:  перебор в KNN числа соседей + метрик подсчёта расстояний. Классификация - ??? Регресссия - ??? 

---------------------
Другие идеи:

- Даны параметры дерева - разобраться что они означают
- Случайный лес - построить прогнозы и усреднить их
- Даны картинки с переобучением и без - ответить галочками где оно есть и пояснить почему словами. Модель + задача + картинка
- Деревья - по картинке понять какие выбраны значения для гиперпараметров - Какая-нибудь задача про метрики машинного обучения и деньги. Есть большой прирост рок-аук, но маленький для продукта. И наоборот маленький в рок-аук и большой для продукта. 

----------------------



