%!TEX TS-program = xelatex
\documentclass[12pt, a4paper, oneside]{article}

\usepackage{amsmath,amsfonts,amssymb,amsthm,mathtools}  % пакеты для математики

\usepackage[english, russian]{babel} % выбор языка для документа
\usepackage[utf8]{inputenc} % задание utf8 кодировки исходного tex файла
\usepackage[X2,T2A]{fontenc}        % кодировка

\usepackage{fontspec}         % пакет для подгрузки шрифтов
\setmainfont{Linux Libertine O}   % задаёт основной шрифт документа

\usepackage{unicode-math}     % пакет для установки математического шрифта
\setmathfont[math-style=upright]{[Neo Euler.otf]} % шрифт для математики

%%%%%%%%%% Работа с картинками %%%%%%%%%
\usepackage{graphicx}                  % Для вставки рисунков
\usepackage{graphics}
\graphicspath{{images/}{pictures/}}    % можно указать папки с картинками
\usepackage{wrapfig}                   % Обтекание рисунков и таблиц текстом

\usepackage{booktabs}
%%%%%%%%%%%%%%%%%%%%%%%% Графики и рисование %%%%%%%%%%%%%%%%%%%%%%%%%%%%%%%%%
\usepackage{tikz, pgfplots}  % язык для рисования графики из latex'a

%%%%%%%%%% Гиперссылки %%%%%%%%%%
\usepackage{xcolor}              % разные цвета

\usepackage{hyperref}
\hypersetup{
	unicode=true,           % позволяет использовать юникодные символы
	colorlinks=true,       	% true - цветные ссылки, false - ссылки в рамках
	urlcolor=blue,          % цвет ссылки на url
	linkcolor=red,          % внутренние ссылки
	citecolor=green,        % на библиографию
	pdfnewwindow=true,      % при щелчке в pdf на ссылку откроется новый pdf
	breaklinks              % если ссылка не умещается в одну строку, разбивать ли ее на две части?
}


\usepackage{todonotes} % для вставки в документ заметок о том, что осталось сделать
% \todo{Здесь надо коэффициенты исправить}
% \missingfigure{Здесь будет Последний день Помпеи}
% \listoftodos --- печатает все поставленные \todo'шки

\usepackage[paper=a4paper, top=20mm, bottom=15mm,left=20mm,right=15mm]{geometry}
\usepackage{indentfirst}       % установка отступа в первом абзаце главы

\usepackage{setspace}
\setstretch{1.15}  % Межстрочный интервал
\setlength{\parskip}{4mm}   % Расстояние между абзацами
% Разные длины в латехе https://en.wikibooks.org/wiki/LaTeX/Lengths


\usepackage{xcolor} % Enabling mixing colors and color's call by 'svgnames'

\definecolor{MyColor1}{rgb}{0.2,0.4,0.6} %mix personal color
\newcommand{\textb}{\color{Black} \usefont{OT1}{lmss}{m}{n}}
\newcommand{\blue}{\color{MyColor1} \usefont{OT1}{lmss}{m}{n}}
\newcommand{\blueb}{\color{MyColor1} \usefont{OT1}{lmss}{b}{n}}
\newcommand{\red}{\color{LightCoral} \usefont{OT1}{lmss}{m}{n}}
\newcommand{\green}{\color{Turquoise} \usefont{OT1}{lmss}{m}{n}}

\usepackage{titlesec}
\usepackage{sectsty}
%%%%%%%%%%%%%%%%%%%%%%%%
%set section/subsections HEADINGS font and color
\sectionfont{\color{MyColor1}}  % sets colour of sections
\subsectionfont{\color{MyColor1}}  % sets colour of sections

%set section enumerator to arabic number (see footnotes markings alternatives)
\renewcommand\thesection{\arabic{section}.} %define sections numbering
\renewcommand\thesubsection{\thesection\arabic{subsection}} %subsec.num.

%define new section style
\newcommand{\mysection}{
	\titleformat{\section} [runin] {\usefont{OT1}{lmss}{b}{n}\color{MyColor1}} 
	{\thesection} {3pt} {} } 


%	CAPTIONS
\usepackage{caption}
\usepackage{subcaption}
%%%%%%%%%%%%%%%%%%%%%%%%
\captionsetup[figure]{labelfont={color=Turquoise}}

\pagestyle{empty}


%%%%%%%%%% Свои команды %%%%%%%%%%
\usepackage{etoolbox}    % логические операторы для своих макросов

% Все свои команды лучше всего определять не по ходу текста, как это сделано в этом документе, а в преамбуле!

% Одно из применений - уничтожение какого-то куска текста!
\newbool{answers}
% \booltrue{answers}
\boolfalse{answers}

\newbool{addanswers}
\boolfalse{addanswers}

\usepackage{enumitem}
% бульпоинты в списках
\definecolor{myblue}{rgb}{0, 0.45, 0.70}
\newcommand*{\MyPoint}{\tikz \draw [baseline, fill=myblue,draw=blue] circle (2.5pt);}
\renewcommand{\labelitemi}{\MyPoint}

% расстояние в списках
\setlist[itemize]{parsep=0.4em,itemsep=0em,topsep=0ex}
\setlist[enumerate]{parsep=0.4em,itemsep=0em,topsep=0ex}

\begin{document}

\section*{Семинар 4: основы статистики! }

\subsection*{Задача 1}

Коллекционер Настя собрала целых $10$ наблюдений и записала их в табличку. Теперь Настя хочет стать аналитиком и проанализировать таблицу. Помогите ей. 

\begin{center}
	\begin{tabular}{lccc}
		\toprule
		имя & пол  & возраст  & вес  \\ \midrule
		Кхал & м  & $14$ &   $80$  \\
		Санса & ж & $16$ &  $40$  \\
		Мелисандра & ж & $20$ &  $40$   \\
		Эддард & м & $20$ &   $80$ \\
		Сандор & м & $14$ &   $80$ \\
		Миссаедея & ж & $25$ &   $40$\\
		Якен & м & $30$ &   $80$\\
		Теон & ж & $23$ &    $40$\\
		Тирион & м & $22$ &    $80$\\
		Станис & м & $16$  &    $440$\\ \bottomrule
	\end{tabular}	
\end{center}

\begin{enumerate}
	\item[а)] Что такое непрерывная переменная? Что такое категориальная переменная? Какие переменные в табличке относятся к непрерывным? Какие к категориальным?  Приведите ещё примеров непрерывных и категориальных переменных! 
	
	\item[б)]  Найдите долю мужчин и женщин в выборке. Постройте для пола гистограмму. 
	
	\item[в)] Найдите средний возраст и медианный возраст.  Что означают эти числа. В чём они измеряются? 

	\item[г)] Найдите дисперсию возраста. В чём измеряются эта величина? Зачем обычно ищут среднее квадратическое отклонение? Найдите его. 	
	
	\item[д)]  Постройте гистограмму для возраста. Считайте, что ширина одного столбца --- 5 лет. Если человек попадает на правую границу отрезка, он попадает в текущий столбец.  Изобразите на гистограмме среднее, медиану. Как бы вы нарисовали на гистограмме стандартное отклонение? 
	
	\item[е)] Что такое выброс? Есть ли выбросы в возрасте? Есть ли выбросы в весе? Как выглядит выброс на гистограмме? Найдите средний вес и медианный вес. Чем медиана в данном случае лучше, чем среднее?
	
	\item[ж)] Чувствительна ли дисперсия к выбросам?
	
	\item[з)] Что такое мода? Почему использовать её для непрерывных переменных не очень хорошая идея? Найдите моду для имени, пола и возраста.  (уточнить что это дискретная мода и тп) 
	
	\item[и)]  Что такое квантиль? Предложите способ,  борьбы с выбросами, основанный на знании того что такое квантиль... 
\end{enumerate}

\ifbool{answers}{
\textbf{Решение:} 

\begin{enumerate}
	\item[а)] Непрерывная переменная не ограничена каким-то конечным набором значений и может принимать любые числовые значения. Например: цена на квартиру, валютный курс, возраст и т.п. 
	
	Категориальная переменная принимает значения из какого-то фиксированного конечного множества. Например: пол, марка машины и тп.
	
	\item[б)]  В выборке $6$ мужчин и $4$ женщины.  Всего $10$ человек.  Значит доля мужчин $\frac{6}{10} = 0.6$, доля женщин $\frac{4}{10} = 0.4$. Нарисуем гистограмму. По оси $x$ будем откладывать возможные значения для нашей переменной, по оси $y$ насколько часто это значение наблюдается в выборке. 
	
	\begin{center}
	\definecolor{zzttqq}{rgb}{0.6,0.2,0.}
	\definecolor{cqcqcq}{rgb}{0.7529411764705882,0.7529411764705882,0.7529411764705882}
	\begin{tikzpicture}[line cap=round,line join=round,x=1.0cm,y=1.0cm]
	\draw [color=cqcqcq,, xstep=1.0cm,ystep=1.0cm] (-3.,-0.04) grid (5.04,5.96);
	\clip(-3.,-0.04) rectangle (5.04,5.96);
	\fill[line width=2.pt,color=zzttqq,fill=zzttqq,fill opacity=0.10000000149011612] (-1.,1.) -- (-1.,4.) -- (0.,4.) -- (0.,1.) -- cycle;
	\fill[line width=2.pt,color=zzttqq,fill=zzttqq,fill opacity=0.10000000149011612] (1.,1.) -- (1.,3.) -- (2.,3.) -- (2.,1.) -- cycle;
	\draw [->,line width=2.pt] (-1.98,0.42) -- (-2.,5.);
	\draw [->,line width=2.pt] (-2.54,1.) -- (4.,1.);
	\draw [line width=2.pt,color=zzttqq] (-1.,1.)-- (-1.,4.);
	\draw [line width=2.pt,color=zzttqq] (-1.,4.)-- (0.,4.);
	\draw [line width=2.pt,color=zzttqq] (0.,4.)-- (0.,1.);
	\draw [line width=2.pt,color=zzttqq] (0.,1.)-- (-1.,1.);
	\draw [line width=2.pt,color=zzttqq] (1.,1.)-- (1.,3.);
	\draw [line width=2.pt,color=zzttqq] (1.,3.)-- (2.,3.);
	\draw [line width=2.pt,color=zzttqq] (2.,3.)-- (2.,1.);
	\draw [line width=2.pt,color=zzttqq] (2.,1.)-- (1.,1.);
	\draw (-2.5,5.5) node[anchor=north west] {$y$};
	\draw (4.1,1.32) node[anchor=north west] {$x$};
	\draw (-1.5,0.7) node[anchor=north west] {\text{мужчины}};
	\draw (0.6,0.7) node[anchor=north west] {\text{женщины}};
%	\draw [line width=2.pt,dash pattern=on 2pt off 2pt] (-1.,4.)-- (-2.,4.);
%	\draw [line width=2.pt,dash pattern=on 2pt off 2pt] (1.,3.)-- (-2.,3.);
	\draw (-2.5,4.5) node[anchor=north west] {$6$};
	\draw (-2.5,3.5) node[anchor=north west] {$4$};
	\draw (-2.5,2.5) node[anchor=north west] {$2$};
	\draw (-2.5,1.6) node[anchor=north west] {$0$};
	\end{tikzpicture}
	\end{center}
	
	\item[в)]  Найдём средний возраст. Для этого сложим все числа и поделим их на количество наблюдений
	
	\[
	\frac{1}{10} \cdot (14 + 16 + 20 + 20 + 14 + 25 + 30 + 23 + 22 + 16) = 20.
	\]
	
	Средний возраст это $20$ лет.  Формула для подсчёта среднего выглядела как 
	
	\[
	\bar x = \frac{1}{n} (x_1 + \ldots + x_n) = \frac{1}{n} \sum_{i=1}^n x_i.
	\]
	
	Привыкайте к формулам. Они будут часто встречаться вам по жизни.  Чтобы найти медиану нам нужно упорядочить всех людей из выборки по возрасту и посмотреть на середину получившегося ряда. 
	
	\[
	14 \quad 14  \quad 16  \quad 16  \quad {\color{myblue}{20}}  \quad {\color{myblue}{20}}  \quad 22  \quad 23  \quad 25  \quad 30
	\]
	
	У нас в середине находятся сразу два человека. Медианой будет их среднее, то есть $20$ лет. Грубо говоря, половина нашей выборки оказывается слева от этого числа, а вторая справа. Медиана находится в серединке.  Оба числа измеряются в годах и обозначают типичный возраст, который присущ людям из выборки. 
	

	\item[г)]  Дисперсия --- это мера разброса. Она показывает насколько разнообразными могут быть элементы в выборке. Чтобы найти её, нужно посмотреть насколько сильно каждый представитель в выборке отличается от текущего. Величина такого отличия называется отклонением. Предположим, что Алёне $18$ лет. Карине $22$ года. Тогда отклонением для Алёны от среднего возраста будет $18 - 20 = -2$ года. Для Карины отклонением будет $22 - 20 = 2$ года. 
	
	Если просуммировать эти отклонения, мы получим $-2 + 2 = 0$. То есть в выборке нет никакого разброса. Все не отличаются от среднего. Это неправда. Для того, чтобы избежать неправды и жить по правде, отклонения возводят в квадрат. Тогда, мы получаем, что суммарное отклонение будет $(-2)^2 + 2^2 = 4 + 4 = 8$. Посмотрев на такое число мы сразу же поймём, что в выборке есть неоднородность. 
	
	Среднее значение квадратов отклонений от среднего и называется дисперсией. Давайте найдём её. Ещё раз выпишем наши наблюдения: 
	
	\[
	14 \quad 14  \quad 16  \quad 16  \quad 20  \quad 20  \quad 22  \quad 23  \quad 25  \quad 30
	\]
	
	Сначала из каждого вычитаем среднее. Это даст нам вектор 
	
	\[
	-6 \quad -6  \quad -4  \quad -4  \quad 0  \quad 0 \quad 2  \quad 3  \quad 5  \quad 10.
	\]
	
	Теперь возводим все отклонения в квадрат
	
	\[
	36 \quad 36  \quad 16  \quad 16 \quad 0  \quad 0 \quad 4  \quad 9 \quad 25  \quad 100.
	\]
	
	Складываем их! Получается $242$. Остаётся разделить это число на  $10$ (количество наблюдений). Получается, что дисперсия составит $24.2$ квадратных года.  Из-за того, что мы каждое слагаемое возводили в квадрат, дисперсия измеряется в квадратных годах.  
	
	Когда мы умножаем одну сторону квадрата на другую, мы получаем его площадь. Она измеряется в квадратных метрах. Тут похожая ситуация. Мы бы хотели вернутся назад, к обычным годам. Для этого из дисперсии извлекают корень и получают штуку под названием стандартное отклонение.  В нашем случае получится $4.9$ года. 
	
	Здесь нам осталось обсудить пару нюансов. 
	
	\begin{itemize}
		\item  Мы возводим отклонения в квадрат не только для того, чтобы сделать все числа положительными. Попутно мы подчёркиваем, что чем больше отклоняется возраст от среднего, тем это хуже. Так штраф за отклонение в два года составит $4$, а за отклонение в три года, $9$.  С подобной логикой мы ещё встретимся, когда будем обсуждать различные метрики, используемые в машинном обучении. 
		
		\item Часто при подсчёте дисперсии вместо формулы 
		
		\[ 
		\hat \sigma^2 = \frac{1}{n} \sum_{i=1}^n (x_i - \bar x)^2,
		\]
		
		которую использовали мы, используют 
		
		\[ 
		\hat \sigma^2 = \frac{1}{n-1} \sum_{i=1}^n (x_i - \bar x)^2.
		\]
		
		Вторая формула на самом деле корректнее, чем первая.  В питоне используется именно она. У этого есть глубокие причины. В полной мере их вы узнаете в курсе по математической статистике. Мы вкратце скажем об этом ближе к концу курса, когда будем говорить про АБ-тесты. Пока держите это в голове, как вопрос, на который у вас нет ответа. Надеюсь, что это будет как следует мучать вас по ночам и стимулировать ботать. 
		
		\item Если распределение у данных нормальное (что такое нормальное распределение --- отдельный и очень важный вопрос), тогда  большая часть выборки, а именно $69\%$ кучкуется в диапазоне между $\bar x - \hat \sigma$ и $\bar x + \hat \sigma$.  
		
		При этом $95 \%$ выборки находится между $\bar x - 2 \cdot \hat \sigma$ и $\bar x + 2 \cdot \hat \sigma$, а $99.9\%$ выборки находятся между $\bar x - 3 \cdot \hat \sigma$ и $\bar x + 3 \cdot \hat \sigma$. 
		
		Правила таких кучкований называют правилом одной, двух и трёх сигм. Их часто используют для проведения АБ-тестов. Об этом мы поговорим ближе к концу курса. Попомните моё слово. 
	\end{itemize}
	
	
	\item[д)] Отмечаем по оси $x$ каждые $5$  лет, как сказано в условии задачи. Для всех людей, попавших в этот отрезок рисуем столбик высоты равной количеству людей, попавших в отрезок. Если человек попадает в \textbf{правую} границу отрезка, он попадает и в столбик. Например, $20$ --- это правая граница второго отрезка. Все люди, которым $20$ лет попадают во второй столбик. Это просто договорённость о том, что делать на границе. Не более того. 
	
	\begin{center}
	\definecolor{zzttqq}{rgb}{0.6,0.2,0.}
	\definecolor{cqcqcq}{rgb}{0.7529411764705882,0.7529411764705882,0.7529411764705882}
	\begin{tikzpicture}[line cap=round,line join=round,,x=1.0cm,y=1.0cm]
	\draw [color=cqcqcq,, xstep=1.0cm,ystep=1.0cm] (-4.,-2.02) grid (6.04,5.98);
	\clip(-4.,-2.02) rectangle (6.04,5.98);
	\fill[line width=2.pt,color=zzttqq,fill=zzttqq,fill opacity=0.10000000149011612] (-2.,0.) -- (-2.,2.) -- (-1.,2.) -- (-1.,0.) -- cycle;
	\fill[line width=2.pt,color=zzttqq,fill=zzttqq,fill opacity=0.10000000149011612] (-1.,0.) -- (-1.,4.) -- (0.,4.) -- (0.,0.) -- cycle;
	\fill[line width=2.pt,color=zzttqq,fill=zzttqq,fill opacity=0.10000000149011612] (0.,0.) -- (0.,3.) -- (1.,3.) -- (1.,0.) -- cycle;
	\fill[line width=2.pt,color=zzttqq,fill=zzttqq,fill opacity=0.10000000149011612] (1.,0.) -- (1.,1.) -- (2.,1.) -- (2.,0.) -- cycle;
	\draw [->,line width=2.pt] (-3.,-0.44) -- (-3.,5.);
	\draw [->,line width=2.pt] (-3.5,0.) -- (5.,0.);
	\draw [line width=2.pt,color=zzttqq] (-2.,0.)-- (-2.,2.);
	\draw [line width=2.pt,color=zzttqq] (-2.,2.)-- (-1.,2.);
	\draw [line width=2.pt,color=zzttqq] (-1.,2.)-- (-1.,0.);
	\draw [line width=2.pt,color=zzttqq] (-1.,0.)-- (-2.,0.);
	\draw [line width=2.pt,color=zzttqq] (-1.,0.)-- (-1.,4.);
	\draw [line width=2.pt,color=zzttqq] (-1.,4.)-- (0.,4.);
	\draw [line width=2.pt,color=zzttqq] (0.,4.)-- (0.,0.);
	\draw [line width=2.pt,color=zzttqq] (0.,0.)-- (-1.,0.);
	\draw [line width=2.pt,color=zzttqq] (0.,0.)-- (0.,3.);
	\draw [line width=2.pt,color=zzttqq] (0.,3.)-- (1.,3.);
	\draw [line width=2.pt,color=zzttqq] (1.,3.)-- (1.,0.);
	\draw [line width=2.pt,color=zzttqq] (1.,0.)-- (0.,0.);
	\draw [line width=2.pt,color=zzttqq] (1.,0.)-- (1.,1.);
	\draw [line width=2.pt,color=zzttqq] (1.,1.)-- (2.,1.);
	\draw [line width=2.pt,color=zzttqq] (2.,1.)-- (2.,0.);
	\draw [line width=2.pt,color=zzttqq] (2.,0.)-- (1.,0.);
	\draw (-2.4,0) node[anchor=north west] {$10$};
	\draw (-1.4,0) node[anchor=north west] {$15$};
	\draw (-0.4,0) node[anchor=north west] {$20$};
	\draw (0.6,0) node[anchor=north west] {$25$};
	\draw (1.6,0) node[anchor=north west] {$30$};
	% цифры внутри
	\draw (-1.9,1.8) node[anchor=north west] {$14$};
	\draw (-1.9,0.8) node[anchor=north west] {$14$};
	\draw (-0.9, 3.8) node[anchor=north west] {$16$};
	\draw (-0.9, 2.8) node[anchor=north west] {$16$};
	\draw (-0.9,1.8) node[anchor=north west] {$20$};
	\draw (-0.9, 0.8) node[anchor=north west] {$20$};
	\draw (0.1,1.8) node[anchor=north west] {$22$};
	\draw (0.1, 0.8) node[anchor=north west] {$23$};
	\draw (0.1,2.8) node[anchor=north west] {$25$};
	\draw (1.1, 0.8) node[anchor=north west] {$30$};
	\draw (4,-0.2) node[anchor=north west] {возраст};
	\draw (-4, 5.6) node[anchor=north west] {число людей};
	\end{tikzpicture}
	\end{center}

Отлично! Гистограмма готова. Каждого человека, которого мы внесли в тот или иной столбец, мы подписали. Давайте отметим на гистограмме медиану и среднее значение. Как это не странно, они оказываютя в "центре" распределения.  


	\begin{center}
	\definecolor{zzttqq}{rgb}{0.6,0.2,0.}
	\definecolor{cqcqcq}{rgb}{0.75,0.75,0.75}
	\definecolor{qqqqff}{rgb}{0.,0.,1.}
	\begin{tikzpicture}[line cap=round,line join=round,,x=1.0cm,y=1.0cm]
	\draw [color=cqcqcq,, xstep=1.0cm,ystep=1.0cm] (-4.,-2.02) grid (6.04,5.98);
	\clip(-4.,-2.02) rectangle (6.04,5.98);
	\fill[line width=2.pt,color=zzttqq,fill=zzttqq,fill opacity=0.10000000149011612] (-2.,0.) -- (-2.,2.) -- (-1.,2.) -- (-1.,0.) -- cycle;
	\fill[line width=2.pt,color=zzttqq,fill=zzttqq,fill opacity=0.10000000149011612] (-1.,0.) -- (-1.,4.) -- (0.,4.) -- (0.,0.) -- cycle;
	\fill[line width=2.pt,color=zzttqq,fill=zzttqq,fill opacity=0.10000000149011612] (0.,0.) -- (0.,3.) -- (1.,3.) -- (1.,0.) -- cycle;
	\fill[line width=2.pt,color=zzttqq,fill=zzttqq,fill opacity=0.10000000149011612] (1.,0.) -- (1.,1.) -- (2.,1.) -- (2.,0.) -- cycle;
	\draw [->,line width=2.pt] (-3.,-0.44) -- (-3.,5.);
	\draw [->,line width=2.pt] (-3.5,0.) -- (5.,0.);
	\draw [line width=2.pt,color=zzttqq] (-2.,0.)-- (-2.,2.);
	\draw [line width=2.pt,color=zzttqq] (-2.,2.)-- (-1.,2.);
	\draw [line width=2.pt,color=zzttqq] (-1.,2.)-- (-1.,0.);
	\draw [line width=2.pt,color=zzttqq] (-1.,0.)-- (-2.,0.);
	\draw [line width=2.pt,color=zzttqq] (-1.,0.)-- (-1.,4.);
	\draw [line width=2.pt,color=zzttqq] (-1.,4.)-- (0.,4.);
	\draw [line width=2.pt,color=zzttqq] (0.,4.)-- (0.,0.);
	\draw [line width=2.pt,color=zzttqq] (0.,0.)-- (-1.,0.);
	\draw [line width=2.pt,color=zzttqq] (0.,0.)-- (0.,3.);
	\draw [line width=2.pt,color=zzttqq] (0.,3.)-- (1.,3.);
	\draw [line width=2.pt,color=zzttqq] (1.,3.)-- (1.,0.);
	\draw [line width=2.pt,color=zzttqq] (1.,0.)-- (0.,0.);
	\draw [line width=2.pt,color=zzttqq] (1.,0.)-- (1.,1.);
	\draw [line width=2.pt,color=zzttqq] (1.,1.)-- (2.,1.);
	\draw [line width=2.pt,color=zzttqq] (2.,1.)-- (2.,0.);
	\draw [line width=2.pt,color=zzttqq] (2.,0.)-- (1.,0.);
	\draw (-0.4,-0.2) node[anchor=north west] {\color{blue} $20$};
	\draw (4,-0.2) node[anchor=north west] {возраст};
	\draw (-4, 5.6) node[anchor=north west] {число людей};
	\draw [fill=qqqqff] (0.,0.) circle (2pt);
	\end{tikzpicture}
\end{center}

Выше мы обсудили, что стандартное отклонение --- величина, которая описывает вариацию выборки вокруг среднего значения и поговорили про правила сигм. Давайте нарисуем от среднего отступы на сигмы вправо и влево. 

\begin{center}
\definecolor{zzttqq}{rgb}{0.6,0.2,0.}
\definecolor{cqcqcq}{rgb}{0.75,0.75,0.75}
\definecolor{qqqqff}{rgb}{0.,0.,1.}
\begin{tikzpicture}[line cap=round,line join=round,x=1.0cm,y=1.0cm]
	\draw [color=cqcqcq,, xstep=1.0cm,ystep=1.0cm] (-4.,-2.02) grid (6.04,5.98);
\clip(-4.,-2.02) rectangle (6.04,5.98);
\fill[line width=2.pt,color=zzttqq,fill=zzttqq,fill opacity=0.10000000149011612] (-2.,0.) -- (-2.,2.) -- (-1.,2.) -- (-1.,0.) -- cycle;
\fill[line width=2.pt,color=zzttqq,fill=zzttqq,fill opacity=0.10000000149011612] (-1.,0.) -- (-1.,4.) -- (0.,4.) -- (0.,0.) -- cycle;
\fill[line width=2.pt,color=zzttqq,fill=zzttqq,fill opacity=0.10000000149011612] (0.,0.) -- (0.,3.) -- (1.,3.) -- (1.,0.) -- cycle;
\fill[line width=2.pt,color=zzttqq,fill=zzttqq,fill opacity=0.10000000149011612] (1.,0.) -- (1.,1.) -- (2.,1.) -- (2.,0.) -- cycle;
\draw [->,line width=2.pt] (-3.,-0.44) -- (-3.,5.);
\draw [->,line width=2.pt] (-3.5,0.) -- (5.,0.);
\draw [line width=2.pt,color=zzttqq] (-2.,0.)-- (-2.,2.);
\draw [line width=2.pt,color=zzttqq] (-2.,2.)-- (-1.,2.);
\draw [line width=2.pt,color=zzttqq] (-1.,2.)-- (-1.,0.);
\draw [line width=2.pt,color=zzttqq] (-1.,0.)-- (-2.,0.);
\draw [line width=2.pt,color=zzttqq] (-1.,0.)-- (-1.,4.);
\draw [line width=2.pt,color=zzttqq] (-1.,4.)-- (0.,4.);
\draw [line width=2.pt,color=zzttqq] (0.,4.)-- (0.,0.);
\draw [line width=2.pt,color=zzttqq] (0.,0.)-- (-1.,0.);
\draw [line width=2.pt,color=zzttqq] (0.,0.)-- (0.,3.);
\draw [line width=2.pt,color=zzttqq] (0.,3.)-- (1.,3.);
\draw [line width=2.pt,color=zzttqq] (1.,3.)-- (1.,0.);
\draw [line width=2.pt,color=zzttqq] (1.,0.)-- (0.,0.);
\draw [line width=2.pt,color=zzttqq] (1.,0.)-- (1.,1.);
\draw [line width=2.pt,color=zzttqq] (1.,1.)-- (2.,1.);
\draw [line width=2.pt,color=zzttqq] (2.,1.)-- (2.,0.);
\draw [line width=2.pt,color=zzttqq] (2.,0.)-- (1.,0.);
\draw (-0.4,-0.2) node[anchor=north west] {\color{blue} $20$};
\draw (4,-0.2) node[anchor=north west] {возраст};
\draw (-4, 5.6) node[anchor=north west] {число людей};
\draw [->,line width=1.pt] (0.,-1.) -- (1.,-1.);
\draw [->,line width=1.pt] (0.,-1.) -- (-1.,-1.);
\draw [->,line width=1.pt] (0,-1.5) -- (-2,-1.5);
\draw [->,line width=1.pt] (0,-1.5) -- (2,-1.5);
\draw (0.6,-0.2) node[anchor=north west] {$\bar x + \hat \sigma$};
\draw (-1.6,-0.2) node[anchor=north west] {$\bar x - \hat \sigma$};
\draw (1.8,-0.8) node[anchor=north west] {$\bar x + 2 \hat \sigma$};
\draw (-3,-0.8) node[anchor=north west] {$\bar x - 2 \hat \sigma$};
\draw [fill=qqqqff] (0.,0.) circle (2.5pt);
\end{tikzpicture}
\end{center}

	\item[е)] 	В возрасте всё хорошо. В весе есть выброс. Кто-то слишком много ест. Давайте найдём среднее и медиану. Среднее окажется равно $\frac{1000}{10} = 100$. Медиана окажется равна $80$. Видим, что выброс существенно сдвинул среднее значение веса в большую сторону. Из-за этого оно перестало отражать типичный вес человека из выборки. Наше представление о людях оказалось искажено. 

	Медиана в отличие от среднего оказывается нечувствительна к выбросам. Это происходит из-за способа её поиска. Мы упорядочиваем наблюдения по порядку и смотрим на то, какое в середине. Значение выброса никак не участвует в подсчёте медианы и именно из-за этого не искажает её. 
	
	На гистограмме переменным, в которых есть выбросы соответствуют очень длинные хвосты. Мы посмотрим на такие гистограммы на копах. 	
	. 
	\item[ж)]  К несчастью, да.  Когда мы считаем её, мы возводим все разности в квадрат. Грубо говоря, разница между средним и выбросом будет большой. Когда мы возведем её в квадрат и прибавим к дисперсии, она очень сильно увеличится. 
	
	\item[з)] Мы с вами определили моду как самое часто встречаемое значение признака в выборке. Для пола модной будут мужчины. Для веса модой будет $80$. Для возраста модой будет либо $20$ либо $14$.  
	
	Для непрерывных переменных использовать моду в качестве меры типичности довольно глупо. Часто бывает так, что непрерывные признаки довольно близки друг к другу, но немного различаются. Чаще всего моду используют, чтобы охарактеризовать именно категориальные переменные. Смотрят на пару: мода, её частота. 
	
	На самом деле моду можно определить так, чтобы она была корректна и для непрерывных признаков. Обычно говорят, что мода это самое вероятное значение в выборке. И после моду ищут по плотности распределения (грубо говоря, по гистограмме), пытаясь понять какому числу соответствует её самая высокая точка. Но об этом вы узнаете на теории вероятностей. 
	
	\item[и)]  На вопрос что такое квантиль, нам поможет ответить медиана. Мы сказали с вами, что если отсортировать выборку по возрастанию, то в середине у неё окажется медиана. 
	
	\[
	14 \quad 14  \quad 16  \quad 16  \quad {\color{myblue}{20}}  \quad {\color{myblue}{20}}  \quad 22  \quad 23  \quad 25  \quad 30
	\]
	
	Получается, что $50\%$ выборки больше медианы, и $50\%$ выборки меньше медианы. Медиана --- это $50\%$ квантиль. По аналогии можно придумать другие квантили. Например, ниже красным отмечены $30\%$ и $70\%$ квантили: 
	
	\[
	14 \quad 14  \quad {\color{red}16 } \quad 16  \quad 20  \quad 20  \quad 22  \quad {\color{red}{23}}  \quad 25  \quad 30
	\]
	
	Ровно $30\%$ меньше $16$ и $70\%$ больше $16$. И наоборот в случае $23$. Среднее и медиана помогают понять какие представители типичны для середины распределения. Квантили помогают понять какие представители типичны для разных кусков распределения. 

	Как мы выяснили выше, выбросы могут существенным образом искажать наши представления о выборке. От них нужно выборку очищать. Один из способов: отрубить все наблюдения, которые находятся выше $99\%$ квантиля и все наблюдения, которы находятся ниже $1\%$ квантиля. Все выбросы такой процедурой будут убиты и мы сможем спокойно работать с выборкой. 

\end{enumerate}
}

\section*{Ещё задачи!} 

Тут находится несколько задачек, о которых вам нужно подумать самостоятельно, в домашних условиях, за чашкой чая.  Возможно, что похожие задачи попадутся вам на самостоятельной работе. Рискнёте проверить?

\subsection*{Задача 2}

Имеется пять чисел: $x$, $9$, $5$, $4$, $7$. При каком значении $x$ медиана будет равна среднему? А можно ли поставить такие цифры в условии задачи, чтобы $x$ не существовал?

\ifbool{answers}{
\textbf{Решение:} 

Расположим числа в порядке возрастания: $4,5,7,9$. В зависимости от расположения $x$ меняется медиана. Так, если мы воткнём $x$ перед или сразу после $4$, медианой будет $5$. Если воткнуть $x$ после $5$, то сам $x$ будет медианой. Если воткнуть $x$ в конце или перед $9$, то медианой окажется $7$. 

Составим три уравнения: 

\begin{equation*} 
\begin{aligned}
& \frac{x + 4 + 5 + 7 + 9}{5} = 5 \Rightarrow x = 0 \\
& \frac{4 + 5 + x + 7 + 9}{5} = x \Rightarrow x = 6.25 \\
& \frac{4 + 5 + 7 + 9 + x}{5} = 7 \Rightarrow x = 10 \\
\end{aligned}
\end{equation*}
}


\subsection*{Задача 3}

Измерен рост $25$ человек. Средний рост оказался равным $160$ см. Медиана оказалась равной $155$ см. Машин рост в $163$ см был ошибочно внесен как $173$ см. Как изменится медиана и среднее после исправления ошибки? А как могут измениться медиана и среднее, если рост Маши равен $153$?

\ifbool{answers}{
\textbf{Решение:} 

Если рост Маши ошибочно был внесен как $173$ см вместо $163$, то при исправлении ошибки изменения никак не отразятся на медиане, потому что ошибочно внесенный рост и ее рост больше медианы. Средний рост уменьшится. В случае $153$ изменения могут коснуться как среднего, так и медианы. 
}

\subsection*{Задача 4}

Деканат утверждает, что если студента $N$ перевести из группы $A$ в группу $B$,то средний рейтинг каждой группы возрастет. Возможно ли такое?

\ifbool{answers}{
\textbf{Решение:} 

Да, возможно. Если средняя оценка $N$ ниже средней оценки группы $A$, но выше средней в группе $B$, то после смены студентом группы средняя оценка каждой группы и ее рейтинг возрастут.

Например, группы $А$ и $В$ состоят из $3$ человек, которые имеют оценки $8$, $9$, $10$ и $1$, $2$, $3$ соответственно. Студент $N$, имеющий оценку $8$, желает перейти в группу $В$. Тогда изменения оценки группы $A$:  $\frac{(9+10)}{2} - \frac{(9+10+8)}{3} = 0.5$, то есть рейтинг группы повысится. Изменения для группы $B$:   $\frac{(1+2+3+8)}{4} - \frac{(1+2+3)}{3} = 1.5$, а значит рейтинг группы $B$ тоже повысится.
}

\subsection*{Задача 5} 

Иногда в качестве меры разброса используют размах. Находят максимальное значение в выборке, минимальное значение выборке, а после вычитают из максимума минимум. Как думаете, такая мера чувствительна к выбросам? Предложите способ сделать её устойчивой к ним.

\ifbool{answers}{
\textbf{Решение:} 

Да, будет. Для того, чтобы сделать эту мену устойчивой к выбросам, можно считать интерквантильный размах, то есть вычитать из $75\%$ квантиля $25\%$ квантиль.
}


\end{document}
